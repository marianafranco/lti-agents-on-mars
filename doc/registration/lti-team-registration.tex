\documentclass{llncs}

\begin{document}
\title{Multi-Agent Programming Contest 2012\\Participation Registration}
\author{}
\institute{}
\maketitle

\begin{abstract}
  Please follow the given template structure for your submission by
  answering the questions as concisely as possible, not exceeding the
  total of \textbf{4} pages. It is vital to explain in this submission
  how are you using a multiagent approach.

%   a \textit{description} of the system, the
%   methodology/tools/infrastructure used and the (team) strategy that
%   you plan to use in the contest.

\end{abstract}


\section*{Introduction}

The LTI-USP, located at the University of S\~ao Paulo, is the oldest and
one of the most relevant research groups in multi-agent systems in Brazil.
The group participated in the 2010 edition of the Multi-Agent Programming Contest
and the cows-and-cowboys scenario was used during the last two years in the Multi-Agent
course held by Jaime Sim\~ao Sichman and Anarosa Alves Franco Brand\~ao at the
Department of Computer Engineering and Digital Systems of the University of S\~ao Paulo.

For this year's tournament, the LTI-USP team is working since the beginning of May, having
invested (approximately) 200 hours. The LTI-USP team if formed by:

\begin{itemize}
\item Team members:
	\begin{itemize}
	\item Mariana Ramos Franco \footnote[1]{team's main-contact} - MSc. Student at University of S\~ao Paulo
	\item Luciano Rosset - Undergraduate Student at University of S\~ao Paulo	
	\end{itemize}
\item Supervisor:
	\begin{itemize}
	\item Jaime Sim\~ao Sichman - Associated Professor at University of S\~ao Paulo\\
	\end{itemize}
\end{itemize}



\section*{System Analysis and Design}

\begin{enumerate}
  %\item How is your system  specified and designed?
  \item Briefly, what is the main strategy of the team?
  \item Will you use any existing multi-agent system
  methodology such as Prometheus, O-MaSE, or Tropos?
 \item Do you plan to distribute your agents on several machines?
 \item Is your solution based on the centralisation of
   coordination/information on a specific agent? Conversely if you
   plan a decentralised solution, which strategy do you plan to use?
\item Describe the communication strategy in the agent team. Can you
   estimate the communication complexity in your approach?
 \item Describe the team coordination strategy (if any)
 \item How are the following agent features implemented:
   \emph{autonomy}, \emph{proactiveness}, \emph{reactiveness}?
% Is your system a truly \textbf{multi}-agent system or
%   rather a centralised system in disguise?
\end{enumerate}



\section*{Software Architecture}

\begin{enumerate}
\item Which programming language do you plan to use to implement the
  multi-agent system? (e.g. 2APL, Jason, Jadex, JIAC, Java, ...)
% \item How would you map the designed architecture (both multi-agent
%   and individual agent architectures) to programming codes, i.e., how
%   would you implement specific agent-oriented concepts and designed
%   artifacts using the programming language?
\item Which development platform and tools are you planning to use?
\item Which runtime platform and tools are you planning to use?
  (e.g. Jade, AgentScape, simply Java, ....)
\item Which algorithms will be used?
\end{enumerate}
Please explain the reasons for your answers.

\end{document}

